\setuppapersize[A4]

\usemodule[microtype]

\setupexternalfigures[location=local,directory=.,conversion=pdf] % lowres,prefix=lowres/]

\definefontfamily [bfont] [serif] [Maiola] [onum=yes,liga=1yes,kern=yes]
\definefontfamily [bfont] [mono] [Source Code Pro] [scaled=0.6]

\definefallbackfamily [bfont] [math] [Maiola] [math:lowercaseitalic] [force=yes]
\definefallbackfamily [bfont] [math] [Maiola] [math:digitsnormal]    [force=yes]
\definefontfamily     [bfont] [math] [TeX Gyre Pagella Math]

\definefontfamily [hfont] [serif] [Maiola] [onum=yes,liga=yes,case=yes,kern=yes,dlig=yes]

\definefontfamily [sufont] [serif] [Maiola Book] [onum=yes,sups=yes]

\definefontfamily [tfont] [serif] [Maiola Book] [tnum=yes,zero=yes]

\setupbodyfont [bfont,10pt]

\setupfontsynonym [Serif] [handling=pure]

\definefontfeature	[default][default][protrusion=quality]

\definefont[secfont][\classfont{hfont}{SerifCaps} at 16pt]
\setuphead[section][style=secfont]

\definefont[titfont][\classfont{hfont}{SerifCaps} at 24pt]
\setuphead[title][style=titfont]

\setuplayout[backspace=20mm,
    width=160mm,
    topspace=20mm,
    header=0mm, 
    footer=0mm,
    height=250mm]

\setupwhitespace[medium]
\setuphead[section][number=no]
\setuphead[title][page=no]

%\startcolumns[n=2]

\chardef\characteralignmentmode=2

\starttext


\definelayer[logo]

\setlayer[logo][x=15cm,y=0cm]{\externalfigure[foodhackingbase-logo.svg][frame=off]}

\flushlayer[logo]


\title{Kombucha Brewing}

\section{Kombucha Manual - Short Workshop Form}

The kombucha culture (SCOBY) is likely to originate in Southeast Asia. To prepare a kombucha beverage you make strong black tea, sweeten it with sugar and add the kombucha mother, letting it ferment for one to two weeks. The fermentation is aerobic and develops well at temperatures between 15-25°C.

\blank[1*line]

\bTABLE

\setupTABLE[c][2,3][alignmentcharacter={.},aligncharacter=yes,align=middle]
\setupTABLE[c][each][frame=off]
\setupTABLE[r][1,2,3][topframe=on]
\setupTABLE[r][last][bottomframe=on]

\bTABLEhead

\bTR
\bTH {\setff{smallcaps}ingredient} \eTH
\bTH {\setff{smallcaps}amount} \eTH
\bTH {\setff{smallcaps}scaling} \eTH
\bTH {\setff{smallcaps}procedure} \eTH
\eTR

\eTABLEhead


\bTABLEbody
\bTR
  \bTC water \eTC
  \bTC 200\,ml \eTC
  \bTC 20\,\% \eTC
  \bTC bring to a boil
  \eTC
\eTR

\bTR
  \bTC black tea \eTC
  \bTC 5\,g \eTC
  \bTC 0.5\,\% \eTC
  \bTC add the tea to the water, let it steep for 5 minutes and strain into fermentation vessel\eTC
\eTR

\bTR
 \bTC light brown sugar \eTC
 \bTC 60\,g \eTC
 \bTC 0.6\,\%\eTC
 \bTC add the sugar to the tea, dissolve well\eTC
\eTR

\bTR
  \bTC cold water \eTC
  \bTC 800\,ml \eTC
  \bTC 80\,\% \eTC
  \bTC[nr=2] dilute the hot tea with the cold water and add the kombucha culture to the mix \eTC 
\eTR

\bTR
  \bTC komubcha SCOBY \eTC
  \bTC 20-100\,g \eTC
  \bTC 2-10\,\% \eTC
\eTR

\bTR
  \bTC \eTC
  \bTC \eTC
  \bTC \eTC
  \bTC close the container, so no insects can get in, but don't seal it since the fermentation is aerobic. don't expose the fermentation to direct sunlight
  \eTC
\eTR

\eTABLEbody

\eTABLE

\blank[1*line]

The brew should be tasted regularly, starting at four days into the fermenation. Once it starts to get sour, you decide when to harvest it, depending on your flavour preference. If you like to add flavour to your drink, you should do so before the secondary fermentation, which makes the drink fizzy. However you can skip secondary fermentation too.

\section{Secondary Fermentation}

\startitemize[n,fit][start=1,stopper={.\space}]
\item taste the beverage and as soon as it is close to your desired flavour, bottle it. The flavour should be bit sweeter than desired because it will become more sour later on.
\item transfer the SCOBY to a new batch (or store refrigerated in fresh medium) and strain the brew through straining bag to remove small pieces of SCOBY
\item pour the brew into bottles using a funnel, leaving around 10\% of each bottle empty
\item seal with a screwcap, label the bottle and let it ferment at 20-25°C for one to two days
\item move the bottles to the fridge and let them age for a week or two
\stopitemize

\section{Flavouring}

If you like you kombucha flavoured, there are various teas you can use. You can make an infusion of jasmine, adding 5 g to 300 ml of boiling water, steeping it for 10 minutes. Strain the tea and add it to the brew, mixing properly before bottling.


\stoptext
